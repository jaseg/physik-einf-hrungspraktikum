\documentclass[12pt,a4paper,notitlepage]{article}
\usepackage[utf8]{inputenc}
\usepackage[a4paper,textwidth=17cm, top=2cm, bottom=3.5cm]{geometry}
\usepackage{eurosym}
%\usepackage{url}
\usepackage[T1]{fontenc}
%\usepackage{ucs}
\usepackage{ngerman} 
\usepackage{setspace}
%\usepackage{fourier}
\usepackage{amssymb,amsmath}
\usepackage{wasysym}
\usepackage{amsthm}
%\usepackage{marvosym}
\usepackage{tabularx}
\usepackage{multicol}
\usepackage{hyperref}
\usepackage[pdftex]{graphicx,color}
%\usepackage{todo}
\definecolor{p-green}{rgb}{0.12,0.57,0.11}
\newcommand{\bitem}{\item[--]}
\newcommand{\litem}[2]{\item[#1 --] #2}
\newcommand{\blitem}[3]{\item[#1 --] \texttt{#2} -- #3}
\newcommand{\gfo}{\grqq\ }
\newcommand{\gfu}{\glqq}
\newcommand{\zquote}[2]{\glqq #1\grqq\ (Z.\ #2)}
\newcommand{\pquote}[1]{\glqq #1\grqq}
\newcommand{\nquote}[2]{#1: \glqq #2\grqq}
\newcommand{\nwquote}[3]{#1 -- \emph{#2}: \glqq #3\grqq}
\newcommand{\nwyquote}[4]{#1 -- \emph{#2} (#3): \glqq #4\grqq}
\newcommand{\diff}{\mathrm{d}}
\renewcommand{\abstractname}{}
\definecolor{orange}{rgb}{1,0.6,0}
\definecolor{d-green}{rgb}{0,0.8,0}
\definecolor{pink}{rgb}{1,0,0.6}
\newcommand{\annot}[1]{\textcolor{red}{#1}}
\newcommand{\ecolor}[1]{\textcolor{pink}{#1}}
\newcommand{\aufgabe}[1]{\section*{\setcounter{section}{#1}Aufgabe #1}}
\newcommand{\re}{\text{Re}}
\newcommand{\im}{\text{Im}}
\onehalfspacing
\setlength{\parskip}{8pt plus4pt minus4pt}
\title{Versuchsprotokoll F5 Dichte fester Körper\\
\small\emph{This document to be found in teh internetz!}\\
\url{https://github.com/jaseg/physik-einf-hrungspraktikum}}
\author{Sebastian Götte, 546408\\
Partner: Erik Lehmann, 546031\\
\emph{und eine weitere Person}}
\date{11-01-04}
\begin{document}
\maketitle
\section{Auswertung}
Die Messdaten befinden sich in der Tabelle. Die zur Ermittlung derselben verwendeten Formeln sind ebenfalls dort zu finden. Die Berechnung der Ableitungen der Dichtefunktion nach Gl. 5 erfolgte mit Maxima und ist in der Datei \texttt{maxima-screencap} zu finden.

Der Vergleich der ermittelten Werte mit Dichtewerten aus der deutschsprachigen Wikipedia ergibt, dass der Wert für Aluminium mit 2.7 exakt mit dem Referenzwert übereinstimmt. Die ermittelte Dichte des Kupfers liegt um einen Betrag jenseits der Messunsicherheit ($0.1\cdot 10^3\frac{kg}{m^3}$ bei $u=0.03\frac{kg}{m^3}$) über dem Referenzwert. Das kann man z.T. durch unterschiedliche Messbedingungen erklären (v.A. die Temperatur des Kupfers), zum Teil durch offenbar fälschlicherweise vernachlässigte oder falsch eingeschätzte Messfehler und ein zu kleines Probenset.
\section{Messdaten}
\subsection{Ermittlung der Ableitungen der Dichtefunktion}
\begin{verbatim}
(%i1) eq:(m1*pw-(m3-m2)*pl)/(m1-(m3-m2));
                             m1 pw - (m3 - m2) pl
(%o1)                        --------------------
                                - m3 + m2 + m1
(%i2) diff(eq, m1);
                           pw         m1 pw - (m3 - m2) pl
(%o2)                -------------- - --------------------
                     - m3 + m2 + m1                    2
                                       (- m3 + m2 + m1)
(%i3) diff(eq, m2);
                           pl         m1 pw - (m3 - m2) pl
(%o3)                -------------- - --------------------
                     - m3 + m2 + m1                    2
                                       (- m3 + m2 + m1)
(%i4) diff(eq, m3);
                     m1 pw - (m3 - m2) pl         pl
(%o4)                -------------------- - --------------
                                      2     - m3 + m2 + m1
                      (- m3 + m2 + m1)
\end{verbatim}
\end{document}
