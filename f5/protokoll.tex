\documentclass[12pt,a4paper,notitlepage]{article}
\usepackage[utf8]{inputenc}
\usepackage[a4paper,textwidth=17cm, top=2cm, bottom=3.5cm]{geometry}
\usepackage{eurosym}
%\usepackage{url}
\usepackage[T1]{fontenc}
%\usepackage{ucs}
\usepackage{ngerman} 
\usepackage{setspace}
%\usepackage{fourier}
\usepackage{amssymb,amsmath}
\usepackage{wasysym}
\usepackage{amsthm}
\usepackage{gensymb}
%\usepackage{marvosym}
\usepackage{tabularx}
\usepackage{multicol}
\usepackage{hyperref}
\usepackage[pdftex]{graphicx,color}
%\usepackage{todo}
\definecolor{p-green}{rgb}{0.12,0.57,0.11}
\newcommand{\bitem}{\item[--]}
\newcommand{\litem}[2]{\item[#1 --] #2}
\newcommand{\blitem}[3]{\item[#1 --] \texttt{#2} -- #3}
\newcommand{\gfo}{\grqq\ }
\newcommand{\gfu}{\glqq}
\newcommand{\zquote}[2]{\glqq #1\grqq\ (Z.\ #2)}
\newcommand{\pquote}[1]{\glqq #1\grqq}
\newcommand{\nquote}[2]{#1: \glqq #2\grqq}
\newcommand{\nwquote}[3]{#1 -- \emph{#2}: \glqq #3\grqq}
\newcommand{\nwyquote}[4]{#1 -- \emph{#2} (#3): \glqq #4\grqq}
\newcommand{\diff}{\mathrm{d}}
\renewcommand{\abstractname}{}
\definecolor{orange}{rgb}{1,0.6,0}
\definecolor{d-green}{rgb}{0,0.8,0}
\definecolor{pink}{rgb}{1,0,0.6}
\newcommand{\annot}[1]{\textcolor{red}{#1}}
\newcommand{\ecolor}[1]{\textcolor{pink}{#1}}
\newcommand{\aufgabe}[1]{\section*{\setcounter{section}{#1}Aufgabe #1}}
\newcommand{\re}{\text{Re}}
\newcommand{\im}{\text{Im}}
\onehalfspacing
\setlength{\parskip}{8pt plus4pt minus4pt}
\title{Versuchsprotokoll F5 Dichte fester Körper\\
\small\emph{This document to be found in the internet}\\
\url{https://github.com/jaseg/physik-einf-hrungspraktikum}}
\author{Sebastian Götte, 546408\\
Partner: Erik Lehmann, 546031 und Markus Hube}
\date{12-01-18}
\begin{document}
\maketitle
\section{Aufgabenstellung und Vorbetrachtungen}
Bestimme die Dichte metallischer Probekörper mit einem Pyknometer und einer Waage. Ermittle die Messunsicherheiten.
\section{Aufbau und Durchführung}
Der Versuchsaufbau ist vorgegeben und besteht aus einer Analysewaage, Metallproben, einem Pyknometer incl. Pinzette und einer Flasche destillierten Wassers. Die Durchführung erfolgt nach den Aufgaben auf dem Aufgabenblatt.
\section{Auswertung}
Das elektronische Messdatenprotokoll findet sich in Anhang \ref{messdatenprotokoll}.

\section{Aufgaben 1,2,3}
Der Mittelwert von $n$ Messwerten wird nach $\overline n=\frac{1}{n}\cdot\sum_{i=1}^nx_i$ berechnet.
\subsection{Ergebnis Aufgabe 1}
Wägung der Probekörper
\subsubsection{Kupfer}
\begin{equation}
\overline m=8.9939\mathrm{g}
\end{equation}
\subsubsection{Alu}
\begin{equation}
\overline m=2.7506\mathrm{g}
\end{equation}
\subsection{Ergebnis Aufgabe 2}
Masse des vollen Pyknometers (ohne Probekörper)
\begin{equation}
\overline m=51.5760\mathrm{g}
\end{equation}
\subsection{Ergebnis Aufgabe 3}
Masse des vollen Pyknometers mit Probe
\subsubsection{Kupfer}
\begin{equation}
\overline m=53.3146\mathrm{g}
\end{equation}
\subsubsection{Alu}
\begin{equation}
\overline m=59.5562\mathrm{g}
\end{equation}

\section{Aufgabe 4}
Formel zur Dichtebestimmung:
\begin{equation}
\label{dichteformel}
\rho = \frac{m_\text{Probe in Luft}\cdot\rho_\text{Wasser}-\left(m_\text{Pyknometer mit Probe}-m_\text{Pyknometer ohne Probe}\right)\cdot\rho_\text{Luft}}{m_\text{Probe in Luft}-\left(m_\text{Pyknometer mit Probe}-m_\text{Pyknometer ohne Probe}\right)}
\end{equation}
Das Pyknometer ist hier jeweils Wassergefüllt und der Auftrieb der Proben ist in den Messwerten enthalten.
\subsection{Ergebnis für Kupfer}
\begin{equation}
\rho=8843.1756\frac{\mathrm{kg}}{\mathrm{m}^3}
\end{equation}
\subsection{Ergebnis für Alu}
\begin{equation}
\rho=2709.7934\frac{\mathrm{kg}}{\mathrm{m}^3}
\end{equation}

\section{Aufgabe 6}
\subsection{Fehlerfortpflanzung}
Die Fehlerfortpflanzung erfolgt durch das Gauß'sche Fehlerfortpflanzungsgesetz nach der folgenden Formel:
\begin{equation}
\label{gaussianerrordings}
{u_y}=\sqrt {\left (\frac{\partial y}{\partial x_1} \cdot u_1 \right)^2 +\left (\frac{\partial y}{\partial x_2} \cdot u_2 \right)^2 +\cdots }
\end{equation}
(aus Wikipedia (de): \url{http://de.wikipedia.org/wiki/Fehlerfortpflanzung#Voneinander_unabh.C3.A4ngige_fehlerbehaftete_Gr.C3.B6.C3.9Fen}, Stand: Jan.\ 2012)

Die hier einzusetzenden Ableitungen habe ich mit Maxima berechnet. Gleichung \ref{dichteformel} sieht in Maxima-Notation folgendermaßen aus:
\begin{verbatim}
(%i1) eq:(m1*pw-(m3-m2)*pl)/(m1-(m3-m2));
                             m1 pw - (m3 - m2) pl
(%o1)                        --------------------
                                - m3 + m2 + m1
\end{verbatim}
Die Ableitung $\rho_{m_1}$ von $\rho$ (aus Gl.\ \ref{dichteformel}) sieht in dieser Notation wie folgt aus:
\begin{verbatim}
(%i2) diff(eq, m1);
                           pw         m1 pw - (m3 - m2) pl
(%o2)                -------------- - --------------------
                     - m3 + m2 + m1                    2
                                       (- m3 + m2 + m1)
\end{verbatim}
Ableitung $\rho_{m_2}$: 
\begin{verbatim}
(%i3) diff(eq, m2);
                           pl         m1 pw - (m3 - m2) pl
(%o3)                -------------- - --------------------
                     - m3 + m2 + m1                    2
                                       (- m3 + m2 + m1)
\end{verbatim}
Ableitung $\rho_{m_3}$:
\begin{verbatim}
(%i4) diff(eq, m3);
                     m1 pw - (m3 - m2) pl         pl
(%o4)                -------------------- - --------------
                                      2     - m3 + m2 + m1
                      (- m3 + m2 + m1)
\end{verbatim}
Die Messunsicherheit wurde anhand der Standardabweichung der Messwerte ermittelt, da sie bei den Wasser involvierenden Messungen wesentlich gröber als die Messgenauigkeit der Waage ist. Die Standardabweichungen wurden mit der LibreOffice-Funktion \texttt{STDEVP} ermittelt. Die daraus berechneten Messunsicherheiten sind:
\begin{description}
\item[Kupferprobe]$u=6.83\cdot10^{-5}\mathrm{g}$
\item[Aluprobe]$u=2.04\cdot10^{-5}\mathrm{g}$
\item[Volles Pyknometer]$u=2.97\cdot10^{-3}\mathrm{g}$
\item[Volles Pyknometer mit Kupferprobe]$u=5.52\cdot10^{-3}\mathrm{g}$
\item[Volles Pyknometer mti Aluprobe]$u=8.21\cdot10^{-4}\mathrm{g}$
\end{description}
Setzt man nun diese Ergebnisse in Gleichung \ref{gaussianerrordings} ein, erhält man für die Messunsicherheit der ermittelten Dichtewerte:
\begin{description}
\item[Kupfer]$u=27\frac{\mathrm{kg}}{\mathrm{m}^3}$
\item[Alu]$u=17\frac{\mathrm{kg}}{\mathrm{m}^3}$
\end{description}

\subsection{Vergleich der Ergebnisse mit Referenzwerten, Fehlerbetrachtung}
Der Vergleich der ermittelten Werte mit Dichtewerten aus der deutschsprachigen Wikipedia ergibt, dass der Wert für Aluminium mit $2.7\cdot10^3\frac{\mathrm{kg}}{\mathrm{m}^3}$ exakt mit dem Referenzwert übereinstimmt. Die ermittelte Dichte des Kupfers liegt mit ca. $80\frac{\mathrm{kg}}{\mathrm{m}^3}$ um einen Betrag jenseits der Messunsicherheit über dem Referenzwert. Das kann man z.T. durch unterschiedliche Messbedingungen erklären (v.A. die Temperatur des Kupfers -- wir maßen bei $22\degree$, der Referenzwert ist für $20\degree$ angegeben), zum Teil durch offenbar fälschlicherweise vernachlässigte oder falsch eingeschätzte Messfehler und ein zu kleines Probenset. Primärer Faktor dürfte die kleine Samplemenge sein, da sich hier die großen Variationen, die beim Umgang mit Wasser und Analysewaagen entstehen, noch stark im Mittelwert niederschlagen können.

\appendix
\section{Messdatenprotokoll}
\label{messdatenprotokoll}
\begin{verbatim}
Versuchende: Sebastian Götte (546408), Markus Hube <hube@physik.hu-berlin.de>,
Erik Lehmann 

Temperatur 22.0 +/- 0.5
Druck 52+48.6 +/- 0.1
Waage:  Systematisch +/- 0.2mg
        Zufällig +/- 0.1mg
Dichte des Wassers lt. Diagramm: 997.77 kg/m^3

Kupfer
8.9938g
8.9941g
8.9938g
8.9941g
8.9937g
8.9938g

Alu
2.7506
2.7506
2.7506
2.7505
2.7505
2.7505

Pyknometer (voll)
51.5614
51.5776
51.5830
51.5810
51.5803
51.5728

Pyknometer (voll) m. Alu
53.3395
53.3153
53.3200
53.3040
53.3121
53.2965

Pyknometer (voll) m. Kupfer
59.5539
59.5548
59.5552
59.5599
59.5577
59.5559

\end{verbatim}
\end{document}
