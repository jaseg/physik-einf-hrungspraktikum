\documentclass[12pt,a4paper,notitlepage]{article}
\usepackage[utf8]{inputenc}
\usepackage[a4paper,textwidth=17cm, top=2cm, bottom=3.5cm]{geometry}
\usepackage{eurosym}
%\usepackage{url}
\usepackage[T1]{fontenc}
%\usepackage{ucs}
\usepackage{ngerman} 
\usepackage{setspace}
%\usepackage{fourier}
\usepackage{amssymb,amsmath}
\usepackage{wasysym}
\usepackage{amsthm}
%\usepackage{marvosym}
\usepackage{tabularx}
\usepackage{multicol}
\usepackage{hyperref}
\usepackage[pdftex]{graphicx,color}
%\usepackage{todo}
\definecolor{p-green}{rgb}{0.12,0.57,0.11}
\newcommand{\bitem}{\item[--]}
\newcommand{\litem}[2]{\item[#1 --] #2}
\newcommand{\blitem}[3]{\item[#1 --] \texttt{#2} -- #3}
\newcommand{\gfo}{\grqq\ }
\newcommand{\gfu}{\glqq}
\newcommand{\zquote}[2]{\glqq #1\grqq\ (Z.\ #2)}
\newcommand{\pquote}[1]{\glqq #1\grqq}
\newcommand{\nquote}[2]{#1: \glqq #2\grqq}
\newcommand{\nwquote}[3]{#1 -- \emph{#2}: \glqq #3\grqq}
\newcommand{\nwyquote}[4]{#1 -- \emph{#2} (#3): \glqq #4\grqq}
\newcommand{\diff}{\mathrm{d}}
\renewcommand{\abstractname}{}
\definecolor{orange}{rgb}{1,0.6,0}
\definecolor{d-green}{rgb}{0,0.8,0}
\definecolor{pink}{rgb}{1,0,0.6}
\newcommand{\annot}[1]{\textcolor{red}{#1}}
\newcommand{\ecolor}[1]{\textcolor{pink}{#1}}
\newcommand{\aufgabe}[1]{\section*{\setcounter{section}{#1}Aufgabe #1}}
\newcommand{\re}{\text{Re}}
\newcommand{\im}{\text{Im}}
\onehalfspacing
\setlength{\parskip}{8pt plus4pt minus4pt}
\title{Versuchsprotokoll O1 Dünne Linsen\\
\small\emph{This document to be found in teh internetz!}\\
\url{https://github.com/jaseg/physik-einf-hrungspraktikum}}
\author{Sebastian Götte, 546408\\
Partner: Erik Lehmann, 546031}
\date{11-01-04}
\begin{document}
\maketitle
\section{Auswertung}
\subsection{Brennweitenbestimmung mittels Abbildungsgleichung}
\begin{equation}
f=\frac{1}{\frac{1}{g}+\frac{1}{b}}
\end{equation}
Da unser Messwert $b$ nicht von der Linse, sondern vom Gegenstand aus gemessen ist, lautet die Formel auf die Messdaten bezogen
\begin{equation}
f=\frac{1}{\frac{1}{g}+\frac{1}{b'-g}}
\end{equation}
$avg$ bezeichnet das arithmetische Mittel der berechneten Brennweiten.
\begin{verbatim}
Linse 4/1
Messfehler: +/- 0.05cm
g=19.9 b=39.4 f=9.849
g=22.5 b=40.1 f=9.875
g=25.0 b=41.2 f=9.830
g=27.5 b=43.3 f=10.035
g=30.0 b=44.8 f=9.911
g=32.5 b=47.0 f=10.027
avg=9.921 +/- 0.05cm
Linse 4/2
Messfehler: +/- 0.25cm
g=19.9 b=98.0 f=15.859
g=25.0 b=70.0 f=16.071
g=30.0 b=65.0 f=16.154
g=35.0 b=65.5 f=16.298
g=35.0 b=65.5 f=16.298
g=40.0 b=67.5 f=16.296
avg=16.163 +/- 0.25cm
Linse 4/3 ist eine Streulinse
Linse 4/4
Messfehler: +/- 0.05cm
g=40.0 b=105.5 f=24.834
g=42.5 b=102.0 f=24.792
g=45.0 b=100.5 f=24.851
g=47.5 b=100.0 f=24.938
g=50.0 b=100.0 f=25.000
g=55.0 b=101.0 f=25.050
avg=24.911 +/- 0.25cm
\end{verbatim}

\subsection{Brennweitenbestimmung nach Bessel}
\subsubsection{Sammellinsen}
\begin{equation}
f=\frac{l^2-e^2}{4l}
\end{equation}
\begin{verbatim}
Die Messfehlerangaben bezeichnen fortan den Messfehler der Werte x1 und x2, nicht des Wertes e=x2-x1.
Linse 4/4
Messfehler: +/- 0.25cm
l=120.0 e=49.5 f=24.895
l=120.0 e=50.0 f=24.792
l=120.0 e=49.0 f=24.998
l=120.0 e=49.5 f=24.895
l=120.0 e=49.0 f=24.998
avg=24.916 +/- 0.5cm
Linse 4/1
Messfehler: +/- 0.05cm
l=50.0 e=23.2 f=9.809
l=50.0 e=23.1 f=9.832
l=50.0 e=23.2 f=9.809
l=50.0 e=23.0 f=9.855
l=50.0 e=23.2 f=9.809
avg=9.823 +/- 0.1cm
Linse 4/2
Messfehler: +/- 0.05cm
l=70.0 e=18.5 f=16.278
l=70.0 e=18.7 f=16.251
l=70.0 e=18.8 f=16.238
l=70.0 e=18.3 f=16.304
l=70.0 e=18.6 f=16.264
avg=16.267 +/- 0.1cm
Linse 4/3: Streulinse (kein Bild. s.o.)
\end{verbatim}
\subsubsection{Linsenkombinationen}
\begin{verbatim}
System 4/1-4/2
Messfehler: +/- 0.05cm
l=35.0 e=18.3 f=6.358
l=35.0 e=18.0 f=6.436
l=35.0 e=18.0 f=6.436
l=35.0 e=18.0 f=6.436
l=35.0 e=18.0 f=6.436
avg=6.420 +/- 0.1cm
System 4/1-4/3
l=55.0 e=17.6 f=12.342
l=55.0 e=17.7 f=12.326
l=55.0 e=17.7 f=12.326
l=55.0 e=17.7 f=12.326
l=55.0 e=17.8 f=12.310
avg=12.326 +/- 0.1cm
System 4/1-4/4
l=30.0 e=5.7 f=7.229
l=30.0 e=5.7 f=7.229
l=30.0 e=5.8 f=7.220
l=30.0 e=5.8 f=7.220
l=30.0 e=5.8 f=7.220
avg=7.223 +/- 0.1cm
\end{verbatim}
\subsection{Aufgabe 4}
\begin{align}
f&=\frac{1}{\frac{1}{f_1}+\frac{1}{f_2}}\\
f_1&=\frac{1}{\frac{1}{f}-\frac{1}{f_2}}
\end{align}
\begin{verbatim}
System 4/1-4/2
f=6.1475664430
System 4/1-4/4
f=7.0952581323
Linse 4/3:
f=-50.8466720891
\end{verbatim}
Die berechneten Brennweiten der Linsensysteme liegen beide innerhalb eines $3mm$-Radius um die nach der Bessel-Methode gemessenen. Das liegt durchaus innerhalb der noch zu diskutierenden Toleranz.
\subsection{Aufgabe 5}
Die Bessel-Methode ist die genauere der beiden Messmethoden, da sich bei ihr eine Ungenauigkeit  der Positionierung der Mittelebene der Linse bezogen auf einen Fixpunkt auf der Halterung der Linse, der zur Messung der Abstände als Referenz herangezogen (Ursprung eines Systematischen Fehlers) wird durch Subtraktion der beiden Messwerte aufgehoben wird.
\section{Fehlerbetrachtung}
\section{Messdaten}
\url{https://github.com/jaseg/physik-einf-hrungspraktikum/blob/master/o1/ergebnisse}
\begin{verbatim}
Fehler:
  Ablesefehler
  Ungenauigkeit der Instrumente (+/- 0.5cm)
  Ungenauigkeit beim Einstellen des Bildes (wie angegeben)
  Mondphase
  Das Maßband lügt. (+ 8m)

A1
Linse 4/1
  g=19.9cm
  b=39.4cm

  g=22.5cm
  b=40.1cm

  g=25.0cm
  b=41.2cm

  g=27.5cm
  b=43.3cm

  g=30.0cm
  b=44.8cm

  g=32.5cm
  b=47.0cm

Linse 4/2
  g=19.9cm
  b=98cm

  g=25.0cm
  b=70cm

  g=30.0cm
  b=65cm
  
  g=35.0cm
  b=65.5cm +/- 0.25cm

  g=35.0cm
  b=65.5cm +/- 0.25cm

  g=40.0cm
  b=67.5cm +/-0.25cm

Linse 4/3
  Streulinse weil kein Bild.

Linse 4/4
  g=40.0cm
  b=105.5 +/- 0.25cm
  
  g=42.5cm
  b=102.0cm +/- 0.25cm

  g=45.0cm
  b=100.5cm +/- 0.25cm

  g=47.5cm
  b=100.0 +/- 0.25cm

  g=50.0cm
  b=100.0 +/- 0.25cm

  g=55.0cm
  b=101.0cm +/- 0.25cm

A2
Linse 4/4
Schirmposition l=120.0cm
  x2=84.5cm +/- 0.25cm
  x1=35.0cm +/- 0.25cm

  x2=84.5cm +/- 0.25cm
  x1=34.5cm +/- 0.25cm

  x2=84.0cm +/- 0.25cm
  x1=35.0cm +/- 0.25cm  

  x2=84.5cm +/- 0.25cm
  x1=35.0cm +/- 0.25cm

  x2=84.0cm +/- 0.25cm
  x1=35.0cm +/- 0.25cm

Linse 4/1
Schirmposition l=50.0cm
  x2=35.7cm
  x1=12.5cm

  x2=35.6cm
  x1=12.5cm
  
  x2=35.8cm
  x1=12.6cm

  x2=35.7cm
  x1=12.7cm

  x2=35.7cm
  x1=12.5cm

Linse 4/2
Schimposition l=70.0cm
  x2=43.8cm
  x1=25.3cm

  x2=43.9cm
  x1=25.2cm

  x2=44.0cm
  x1=25.2cm

  x2=43.5cm
  x1=25.2cm

  x2=43.8cm
  x1=25.2cm

Linse 4/3: Streulinse (kein Bild. s.o.)

A3
System 4/1-4/2
Schirmposition l=35.0cm
  x2=26.5cm
  x1=8.2cm

  x2=26.3cm
  x1=8.3cm

  x2=26.4cm
  x1=8.4cm

  x2=26.3cm
  x1=8.3cm

  x2=26.4cm
  x1=8.4cm

System 4/1-4/3
Schirmposition l=55.0cm
  x2=35.6cm
  x1=18.0cm

  x2=35.8cm
  x1=18.1cm

  x2=35.7cm
  x1=18.0cm

  x2=35.7cm
  x1=18.0cm

  x2=35.7cm
  x1=17.9cm


System 4/1-4/4
Schirmposition l=30.0cm
  x2=17.4cm
  x1=11.7cm

  x2=17.5cm
  x1=11.8cm

  x2=17.6cm
  x1=11.8cm

  x2=17.5cm
  x1=11.7cm

  x2=17.5cm
  x1=11.7cm

\end{verbatim}
\end{document}
